
%% start of file 'JohnVanNoteResume.tex'.
%% Copyright 2006-2013 Xavier Danaux (xdanaux@gmail.com).
%%
%% John Van Note
%% 2020-02-02
%%
%% Resume source code
%%

% This work may be distributed and/or modified under the
% conditions of the LaTeX Project Public License version 1.3c,
% available at http://www.latex-project.org/lppl/.


% possible options include font size ('10pt', '11pt' and '12pt'), paper size ('a4paper', 'letterpaper', 'a5paper', 'legalpaper', 'executivepaper' and 'landscape') and font family ('sans' and 'roman')
\documentclass[11pt,a4paper,sans]{moderncv}

% moderncv themes
% style options are 'casual' (default), 'classic', 'oldstyle' and 'banking'
\moderncvstyle{classic}

% color options 'blue' (default), 'orange', 'green', 'red', 'purple', 'grey' and 'black'
\moderncvcolor{black}

% to set the default font; use '\sfdefault' for the default sans serif font, '\rmdefault' for the default roman one, or any tex font name
%\renewcommand{\familydefault}{\sfdefault}

% uncomment to suppress automatic page numbering for CVs longer than one page
\nopagenumbers{}

% character encoding
% if you are not using xelatex ou lualatex, replace by the encoding you are using
\usepackage[utf8]{inputenc}

% if you need to use CJK to typeset your resume in Chinese, Japanese or Korean
%\usepackage{CJKutf8}

% adjust the page margins
\usepackage[scale=.8]{geometry}

% if you want to change the width of the column with the dates
%\setlength{\hintscolumnwidth}{3cm}
% for the 'classic' style, if you want to force the width allocated to your name and avoid line breaks. be careful though, the length is normally calculated to avoid any overlap with your personal info; use this at your own typographical risks...
%\setlength{\makecvtitlenamewidth}{10cm}

% personal data
\name{John}{Van Note}
% optional, remove / comment the line if not wanted
%\title{Resume title}
\address{800 P St. NW, Apt.620}{Washington, DC}{20001}
\phone[mobile]{215.518.0823}
\email{johnlvannote@protonmail.com}
\homepage{https://github.com/JohnVanNote}

\begin{document}
%-----       resume       ---------------------------------------------------------
\makecvtitle

\section{Experience}
\cventry{Mar. 2017 - Current}{Senior Software Engineer, Identity \& Access Management}{Ernst \& Young (acquired from Sila Solutions Group December 2019)}{Arlington, VA}{}{
  \textit{Technologies used: Java, Apache Tomcat, SQL, Vagrant, Git, Jenkins, AWS.}%
  \begin{itemize}%
        \item Lead the design and development of complex integrations with third-party Identity \& Access Management (IAM) solutions, primarily leveraging Java using TestNG/EasyMock for unit testing, and Vagrant/Docker for integration testing.
        \item Designed the high level architecture for a resume parsing application, following an MVC pattern using ReactJS for the front end, Spring Boot for the business logic, ElasticSearch for data storage, and Jenkins for build automation.
        \item Implemented practice-wide code standardization practices including automation, code review methodology, telemetry, and testability.
        \item Mentored and trained over twenty junior software developers including experienced hires.
      \end{itemize}
}

\cventry{Feb. 2016 - Mar. 2017}{Software Support Engineer}{Corporation Service Company}{Wilmington, DE}{}{
  \textit{Technologies used: Java/Groovy, Python, BASH, Apache HTTP Server, Apache Tomcat, IIS, Oracle WebLogic, SQL, Git, VirtualBox.}%
  \begin{itemize}%
        \item Designed and developed a Groovy-based application, which automated a previously manually-compiled monthly application performance index. The application aggregated data from multiple reporting tools leveraging the Splunk SDK for Java and reading from an Oracle 11g database backend.
        \item Engineered and supported the implementation of enterprise ETL and Business Intelligence Reporting tools across four environments, spanning more than fourteen application servers. Participated in planning, configuration, and troubleshooting of all server-side issues.
        \item Developed a solution to remediate application latency in an Apache Tomcat application by creating an extensible custom Python library to execute BASH commands on Linux servers. 
        \item Authored a variety of BASH scripts to common Linux/Solaris server issues.
      \end{itemize}
}


\section{Education}
\cventry{Sep. 2008 - Jun. 2013}{Bachelor of Science in Business Administration, Minor in Computer Science}{Drexel University}{Philadelphia, PA}{\textit{3.59 GPA}}{}

\section{Technical Tools}
\cvlistitem{\textbf{Programming/Scripting Languages}: Java, Groovy, Python, C/C++, BASH, VBA, JavaScript, PowerShell.}
\cvlistitem{\textbf{Development Tools}: Git/SVN, Vagrant, Jenkins, Maven/Ant, IntelliJ/Eclipse, iTerm 2.}
\cvlistitem{\textbf{Web Servers}: Apache HTTP Server, IIS, Apache Tomcat, Oracle WebLogic.}
\cvlistitem{\textbf{Databases}: Oracle, MySQL, Amazon Aurora.}

\end{document}
%% end of file 'JohnVanNoteResume.tex'.
